\documentclass[twocolumn,apj,iop,tighten]{emulateapj2}

\usepackage{float}
\usepackage{graphicx}
\usepackage{xfrac}
\usepackage{adjustbox}
\usepackage{upgreek}


\input{../../tex/defs}

\begin{document}
\title{}
\shortauthors{Berti et al.}

\author{Angela M.\ Berti,\altaffilmark{1}
}
	
\altaffiltext{1}{Department of Physics \& Astronomy, University of Utah, 201 Presidents' Cir, Salt Lake City, UT 84112, USA}
%\altaffiltext{1}{Center for Astrophysics and Space Sciences, Department of Physics, University of California, 9500 Gilman Dr., La Jolla, San Diego, CA 92093, USA}

\begin{abstract}
TODO
\end{abstract}

%%%%%%%%%%%%%%%%%%%%%%%%%%%%%%%%%%%%%%%%
\section{Introduction}\label{sec:intro}
%%%%%%%%%%%%%%%%%%%%%%%%%%%%%%%%%%%%%%%%

halos = parent halos + subhalos
All subhalos are halos, but not all halos are subhalos.

%%%%%%%%%%%%%%%%%%%%%%%%%%%%%%%%%%%%%%%%
\section{Data}\label{sec:data}
%%%%%%%%%%%%%%%%%%%%%%%%%%%%%%%%%%%%%%%%

TODO

\subsection{Photometry and redshift estimates}\label{subsec:photometry}
We use publicly available catalogs from the ninth data release (DR9) of the DESI Legacy Imaging Surveys (Dey+ 2019). The Legacy Surveys provide optical imaging in the $g$, $r$, and $z$ bands is from a combination of three public surveys: the DECam Legacy Survey (DECaLS; ref),
%conducted with the Dark Energy Camera (DECam) on the Blanco 4-meter telescope at the Cerro Tololo Inter-American Observatory in Chile;
the Beijing-Arizona Sky Survey (BASS; ref),
%conducted with the 90Prime camera at the prime focus of the Bok 2.3-meter telescope at Kitt Peak National Observatory (KPNO);
and the Mayall $z$-band Legacy Survey (MzLS; ref).
%, conducted with the MOSAIC-3 camera at the prime focus of the 4-meter Mayall telescope at KPNO.
14,000 square degrees of sky visible from the northern hemisphere
2 contiguous regions within the northern and southern galactic caps


DR9 also includes four mid-infrared bands from the Wide-field Infrared Survey Explorer (WISE; ref), although only the 3.4-$\upmu$m $W1$-band is relevant DESI LRG target selection.


Zhou+ 2020 compute photometric redshifts for the full catalog of DECaLS DR8 objects using a random forest regression machine learning algorithm. They estimate their redshifts to be accurate for objects with apparent $z$-band magnitude $z < 21$, well beyond the $z$-band cut we use to select our target galaxy samples, described in \S\ref{subsec:parent_samples} below.


\subsection{Target galaxy samples}\label{subsec:parent_samples}

One of the primary goals of this work is to develop magnitude-limited mock galaxy catalogs that represent a superset of the color-magnitude space occupied by DESI LRGs.

%%%%%%%%%%%%%%%%%%%%%%%%%%%%%%%%%%%%%%%%
%\setlength\extrarowheight{2pt}
%\setlength{\tabcolsep}{5pt}
\begin{deluxetable}{ r p{6.5cm} }%[h!]
\tablecaption{TODO
\label{tab:masks}
}
\tablehead{
\colhead{ \texttt{MASKBIT}} & \colhead{Description}
}
\startdata
5, 6, 7 & bad pixel in all of a set of overlapping $g$, $r$, or $z$-band images \\
8, 9 & bad pixel in a WISE $W1$ or $W2$ bright star mask \\
11 & pixel within locus of a radius-magnitude relation for Gaia (ref?) DR2 stars to $G < 16$ \\
12 & pixel in a Siena Galaxy Atlas (ref?) large galaxy \\
13 & pixel in a globular cluster \\
\hline \\
\vspace{-4ex} \\
\colhead{\texttt{FITBIT}} & \colhead{Description} \\
\vspace{-2ex} \\
\hline \\
\vspace{-4ex} \\
6 & source is a medium-bright star \\
7 & Gaia source (ref?) \\
8 & Tycho-2 star (ref?) \\
\enddata
\end{deluxetable}
%%%%%%%%%%%%%%%%%%%%%%%%%%%%%%%%%%%%%%%%
For additional details about these masks see legacysurvey.org/dr9/bitmasks/

Our parent galaxy samples are defined by the following cuts:
$z < 20.7$

We also remove stars by excluding catalog sources of \texttt{TYPE = PSF}.

geometric mask to ensure complete angular coverage by available randoms provided with DECaLS catalogs

r-band and W1-band absolute magnitude cuts
	eliminate fainter galaxies (with larger photo-z errors) that don't overlap relevant LRG magnitude range
	smaller parent sample size improves computation time
photo-z redshift bins (nominal range +/- 150 Mpc/h)

Data samples:
	fields: 2 (north, south)
	absolute magnitude cuts: 2 (r, W1)
	photo-z bins: 4-6 ( (0.3,) 0.4, 0.5, 0.6, 0.7, 0.8(, 0.9) )
TOTAL: 2 x 2 x 4(,5,6) = 16(,20,24)

TODO: summary table of parent data samples?



%For each of the above 16(,20,24) data samples:
%	compute full sample (no abs mag bins) wp(rp) (cross-correlation of strict photo-z bin with padded bin)
%		FINAL cuts (0.4 < zphot < 0.5; approx. equal parent sample sizes):
%			Mr < -20.5
%			MW1 < -21.8
%			above are IN PROGRESS for zphot=(0.4,0.5) NORTH
%	compute wp(rp) in TBD absolute magnitude bins (r, W1)
%		2 or 3 bins depending on samples sizes and/or abs mag range
%	compute full sample luminosity function
%	compute full sample effective galaxy number density as function of progressively fainter: neff(< Mx)
%		ng_eff(< Mx) = sum( 1 / (angular_survey_area*Veff) )
%		V_eff = volume corresponding to max redshift at which galaxy would have been observed by survey
%	compute color distributions (g-r, g-W1) in narrow TBD absolute magnitude bins (Mr, MW1)


%%%%%%%%%%%%%%%%%%%%%%%%%%%%%%%%%%%%%%%%
\section{Simulations}\label{sec:sims}
%%%%%%%%%%%%%%%%%%%%%%%%%%%%%%%%%%%%%%%%

MDPL2 (refs)
1 \Gpch periodic box
DM particle resolution
cosmology
available snapshots
Rockstar halo finder (refs)


%%%%%%%%%%%%%%%%%%%%%%%%%%%%%%%%%%%%%%%%
\section{Subhalo abundance matching}\label{sec:sham}
%%%%%%%%%%%%%%%%%%%%%%%%%%%%%%%%%%%%%%%%

Subhalo abundance matching (SHAM; refs) places mock galaxies into halos by exploiting the correlation between some galaxy property\textemdash usually stellar mass or luminosity\textemdash and a halo property such as virial mass or circular velocity, \vcirc, (TODO: elaborate on vcirc options). Here we use SHAM to create a mock galaxy population with the same abundance and luminosity distribution of the target galaxy sample by assuming the following relation:
%
\begin{equation}\label{eq:sham}
\neff(<M)=n_{\rm h}(>\vcirc),
\end{equation}
%
\noindent i.e., the (effective) number density of galaxies, \neff, of magnitude $M$ or brighter equals the number density of halos, $n_{\rm h}$, with circular velocity \vcirc or greater.

TODO: also tested \vmax and \vmaxmpeak and found that \vmaxmpeak works as well as \vpeak, while using \vmax yields significantly worse results.

\subsection{Luminosity assignment}\label{subsec:luminosity_assign}

To assign absolute magnitudes to halos in each redshift bin we implement the following procedure:
%
\begin{enumerate}[leftmargin=0pt, itemindent=24pt, listparindent=10pt, label=(\arabic*), nosep]
\item \label{step:neff}
Compute for each parent galaxy sample the effective galaxy number density as a function of absolute magnitude in band ${X=\{M_r, M_{W1}\}}$, $\neff(<M_X)$. Absolute magnitudes are computed with distance moduli from photometric redshifts, and $K$-corrections are calculated using the \idl package \kcorrect (ref) with DECam $grz$ and WISE $W1$ and $W2$ filter responses. We $K$-correct each target galaxy sample to the redshift of the relevant simulation snapshot, \zsim (TODO: see table?), e.g., galaxies in the ${0.4 < \zphot < 0.5}$ bin are $K$-corrected to $\zsim=0.42531$.

We compute $\neff(<M_X)$ for each parent galaxy sample using the method of inverse effective volumes (TODO: refs). For each galaxy in the sample we calculate an effective volume, $V_{\rm eff}$:
%
\begin{equation}
{V_{\rm eff}(M_X) = f_\Omega \left( V_{\rm max}(M_X)-V_{\rm min} \right)},
\end{equation}
%
\noindent where $f_\Omega$ is the fractional solid angle covered by the parent galaxy sample, and $V_{\rm min}$ is the comoving volume of the lower limit of the redshift bin. $V_{\rm max}(M_X)$ is the comoving volume of either the upper limit of the redshift bin, or of the maximum possible redshift a galaxy of magnitude $M_X$ could have and still be observed at the magnitude limit of the sample, whichever is smaller.

The effective galaxy number density is the sum of the inverse of $V_{\rm eff}(M_X)$ over all galaxies in the sample:
%
\begin{equation}\label{eq:neff}
  \neff(<M_X) = \sum_i \left[ V^i_{\rm eff}(M_X) \right]^{-1}.
\end{equation}
%
\item \label{step:nh}
Assign absolute magnitudes to mock galaxies with no scatter in the luminosity\textendash \vpeak relation according to Equation~\ref{eq:sham}. For each halo we find the cumulative number density $n_{\rm h}(>\vpeak)$ corresponding to its value of \vpeak. We then assign to each halo a mock galaxy with the absolute magnitude $M_X$ at which the effective number density $\neff(<M_X)$ of the parent galaxy sample equals $n_{\rm h}(>\vpeak)$. (TODO: figure?)
%
\item \label{step:sham_scatter}
To incorporate magnitude-dependent scatter (\sigmamag) into the luminosity\textendash \vpeak relation we assign to each mock galaxy a new absolute magnitude, $M'_X$, where $M'_X$ is drawn from a Gaussian distribution centered at $M_X$ with width $\alpha\,M_X$, where $\alpha>0$, i.e., the width of the distribution is proportional to $M_X$. We then assign absolute magnitudes to mock galaxies as in step \ref{step:nh}, but according to $M'_X$ instead of $M_X$. This method preserves both the galaxy luminosity function and the original \vpeak distribution.
%
\item \label{step:los_scatter}
Scatter the positions of mock galaxies along one axis of the simulation volume (we choose the $x$-axis as this ``line-of-sight" axis to avoid confusion with redshift) to mimic the uncertainty in radial position of our target galaxy samples due to photometric redshift errors. For each mock galaxy we draw a ``scattered" $x$-axis coordinate $x'$ from a Gaussian distribution of width $\sigma_x$ centered at the galaxy's original $x$-axis position. Mock galaxies that scatter out of the simulation volume of $1~h^{-3}{\rm Gpc}^3$ are wrapped back in to preserve the periodic boundary conditions, e.g., a mock galaxy at $x=25~\Mpch$ that scatters to $x'=-50~\Mpch$ would be placed at $x=950~\Mpch$.
%
\item \label{step:model_wp}
Compute the projected correlation function, \wprp, of each mock catalog (see \S\ref{subsec:wprp}), and the goodness-of-fit per degree of freedom, \chisqred (see \S\ref{subsec:error}), of the model fit to the projected correlation function of the relevant parent galaxy sample. We measure \chisqred of \wprp in bins of absolute magnitude ($M_r$ or $M_{W1}$; see Figure~\ref{TODO}).
%
\item \label{step:model_refine}
Repeat steps \ref{step:sham_scatter} through \ref{step:model_wp} for additional values of $\alpha$ and $\sigma_x$ as needed to minimize \chisqred in each magnitude bin. In practice we first sparsely sample a wide range of values of $\alpha$ and $\sigma_x$:
%
\begin{subequations}
  \begin{align}
    (0.1 \leq \alpha \leq 0.9),\ &\Delta\alpha=0.1 \\
    (10 \leq \sigma_x \leq 100\, \Mpch),\ &\Delta\sigma_x=10\, \Mpch
  \end{align}
\end{subequations}
%
\noindent We then more densely sample narrower ranges of both parameters around the initial coarse-grained values that minimize \chisqred.
%
\item \label{step:sigma_linear}
Finally, we parameterize the dependence of \sigmamag and \sigmalos on absolute magnitude as follows:
%
\begin{subequations}\label{eq:sigma_linear}
\begin{align}
  \sigmamag(M_X) &= a\, M_X + \sigma_{{\rm mag},0} \\
  \sigmalos(M_X) &= b\, M_X + \sigma_{{\rm los},0}.
\end{align}
\end{subequations}
%
\noindent The values of $(a,\, \sigma_{{\rm mag},0})$ and $(b,\, \sigma_{{\rm los},0})$ are determined by linear fits to $\sigmamag^i$ versus $M^i_X$, and $\sigmalos^i$ versus $M^i_X$, respectively, where $\sigmamag^i$ and $\sigmalos^i$ are the values that minimize \chisqred of \wprp in the $i$th magnitude bin, $M^i_X$. Figure~\ref{TODO} shows the $r$-band and $W1$-band magnitude bins used for each redshift bin.
\end{enumerate}

\subsection{Projected correlation functions}\label{subsec:wprp}

The luminosity assignment stage of our modeling procedure involves just two free parameters, $\alpha$ and \sigmalos, which account for scatter in the luminosity\textendash\vcirc relation and photometric redshift errors of our target galaxy samples, respectively. We constrain these parameters by fitting the projected correlation functions of mock galaxy catalogs created from our model to those of the corresponding parent galaxy samples.

One goal of this paper is to exploit the completeness and enormous volume of DECaLS data, which comes at the expense of the precision clustering measurements achievable with spectroscopic redshifts. Zhou+ 2020 has demonstrated the constraining power of the projected correlation function, \wprp, of DECaLS galaxies computed with line-of-sight distances derived from photometric redshifts (they use this statistic to constrain the halo occupation distribution (HOD, TODO: defined earlier?) parameters of LRGs selected from DECaLS DR7). The projected correlation function integrates the 3D correlation function, \xir, along the line-of-sight, effectively eliminating the effects of radial distance uncertainty due to photometric redshift errors:
%
\begin{equation}
\wprp \equiv \int_{-\pimax}^{\pimax}\!\xi(\rp,\pi)\,d\pi \approx 2\!\int_0^{\pimax}\!\xi(\rp,\pi)\,d\pi,
\end{equation}
%
\noindent where \rp is the projection of $r$ into the plane perpendicular to the line-of-sight distance, $\pi$.

We use the \corrfunc package (ref) to calculate \wprp for both our target galaxy samples and mock catalogs in 19 logarithmic bins between ${\rp>0.04\Mpch}$ and ${\rp<52.8\Mpch}$.

As our data samples are confined to narrow redshift ranges of width ${\Delta z = 0.1}$, photometric redshift errors will cause some galaxies that belong to a given redshift bin to scatter into an adjacent bin and be excluded from the calculation of \wprp for their true bin. To account for this we adopt the method used by Zhou+ 2020 (see their Figure 8): instead of the autocorrelation function of galaxies within each redshift bin, we use the Landy-Szalay (ref) estimator for the cross-correlation of two samples, $D_1$ and $D_2$:
%
\begin{equation}\label{eq:landy-szalay}
\wprp = 2\pimax\!\left(\frac{D_1D_2 - D_1R_2 - D_2R_1}{R_1R_2} + 1 \right).
\end{equation}
%
\noindent Each term of Equation~\ref{eq:landy-szalay} denotes pair counts between two samples, where $D$ and $R$ respectively indicate samples of data (galaxies) and random points with the same angular and redshift distributions as the corresponding data sample. Here $D_1$ is all galaxies within a given redshift bin: ${\zmin < \zphot < \zmax}$, where \zmin and \zmax are the limits of the bin, while $D_2$ is all galaxies within a wider redshift range defined by ${(\zmin-\pimax) < \zphot < (\zmax+\pimax)}$, where ${\pimax=150~\Mpch}$.
We verify our implementation of this method with \corrfunc by reproducing the projected correlation functions of DECaLS LRGs that Zhou+ 2020 obtain (see their Figure 9) using different clustering code.

DECaLS data includes catalogs of random points with the same angular sky coverage and mask information as the survey footprint, which we use to construct our random samples. We use 20 times as many random as data points for each galaxy sample, and draw redshifts for random points from the redshift distribution of the corresponding data sample.

To measure \wprp of our mock catalogs we take advantage of the \corrfunc \texttt{theory} module, which can quickly calculate the autocorrelation function of a sample within a periodic volume using analytic randoms. We confirmed that this method produces the expected result by calculating \wprp of several mock catalogs directly from pair counts between mock galaxies and catalogs of random points constructed for the simulation volume.


\subsection{Jackknife error estimation and goodness-of-fit}\label{subsec:error}

To estimate the uncertainty of the \wprp measurements of our target galaxy samples we use HEALPix (TODO: healpy?) to divide the angular sky coverage of each sample into $N_{\rm jk}$ regions of roughly equal area, suitable for jackknife resampling. We then measure \wprp in each of these $N_{\rm jk}$ regions, and compute the covariance matrix as follows:
%
\begin{equation}\label{eq:cov}
{\rm Cov}_{ij} = \frac{N_{\rm jk}-1}{N_{\rm jk}}
\sum^{N_{\rm jk}}_{\ell=1}\left(\omega^\ell_i-\overline{\omega}_i\right) \left(\omega^\ell_j-\overline{\omega}_j\right),
\end{equation}
%
\noindent where $\omega^\ell_i$ and $\omega^\ell_j$ are \wprp of the $\ell$th jackknife region for the $i$th and $j$th \rp bins, respectively, and $\overline{\omega}_i$ and $\overline{\omega}_j$ are the mean values of \wprp across all jackknife regions for the $i$th and $j$th \rp bins, respectively.

With the covariance matrix in hand we quantify how successful any instance of our model is at fitting the projected correlation function of the data by computing $\chi^2$ per degree of freedom $(\chisqred)$:
%
\begin{equation}\label{eq:chisq}
\chisqred=\frac{1}{\nu}\sum_{i=1}^{N_{r_{\rm p}}}
\sum_{j=1}^{N_{r_{\rm p}}}\left(\omega_i-\omega^{\rm mod}_i\right)
\!\left({\rm Cov}^{-1}\right)_{ij}\!\left(\omega_j-\omega^{\rm mod}_j\right)\!,
\end{equation}
%
\noindent where $N_{\rp}$ is the number of \rp bins, $\nu$ is the number of degrees of freedom minus the number of model parameters (here ${N_{\rp}-2=17}$), $\omega_i$ and $\omega_j$ are the data \wprp values in the $i$th and $j$th \rp bins, respectively, and $\omega^{\rm mod}_i$ and $\omega^{\rm mod}_j$ are the \wprp values of the relevant mock catalog in the $i$th and $j$th \rp bins, respectively.


\subsection{Rest-frame color assignment}\label{subsec:color_assign}

Color matching assumes a correlation between rest-frame galaxy color at fixed absolute magnitude and some proxy for the age of the halo in which each mock galaxy resides, with redder colors generally assigned to older halos. The first step of color assignment is to determine the best proxy for halo age. Previous implementations of this technique at $z\sim0$ using SDSS data have used TODO (refs).
%
\begin{equation}\label{eq:zstarve}
\zstarve \equiv \max\{\zchar,\, \zacc,\, \zform\}.
\end{equation}
%
\noindent In Equation~\ref{eq:zstarve}:

\begin{itemize}
\item \zchar is either the redshift at which a halo's mass first exceeds some characteristic value, \mchar, or the redshift of the relevant simulation snapshot (\zsim) for halos that never achieve \mchar,
%
\item \zacc is the redshift at which a subhalo accretes onto a parent halo (for host halos ${\zacc=\zsim}$), and
%
\item \zform is the ``formation" redshift at which a halo transitions from the fast to slow accretion regime.
\end{itemize}
\noindent We use same definition of \zform as Hearin \& Watson 2013, motivated by Wechsler+ 2002:
%
\begin{equation}\label{eq:zform}
\zform \equiv \frac{c_{\rm vir}}{4.1 a_0} - 1,
%c_{\rm vir} &= \frac{R_{\rm vir}}{R_{\rm s}}\ {\rm at}
%\begin{cases}
%\zsim\ {\rm (host\ halos)} \\
%\zacc\ {\rm (subhalos)}
%\end{cases}, \\
%a_0 &=
%\begin{cases}
%\zsim\ {\rm (host\ halos)} \\
%\zacc\ {\rm (subhalos)}
%\end{cases}.
\end{equation}
%
\noindent where ${c_{\rm vir} = R_{\rm vir}/R_{\rm s}}$ is a halo's concentration at the time of observation, indicated by $a_0$. For host halos $a_0$ is the scale factor of the relevant simulation snapshot, while for subhalos $a_0$ is the scale factor at the time of accretion: ${\zacc = 1/{a_0} - 1}$. $R_{\rm vir}$ is the virial radius of a halo, and $R_{\rm s}$ is the NFW scale radius.

Hearin \& Watson 2013 use ${\mchar=10^{12}~\hmsun}$ TODO...

%\begin{equation}\label{eq:}
%\end{equation}



%Free model parameters (C=continuous, D=discrete):
%	(D) (sub)halo property for SHAM: \vpeak, \vmax, \vmaxmpeak
%	(D) (sub)halo property for age-matching (color assignment): \zstarve, \zform, \zacc, \zchar
%	(C) amplitude and (D) functional form Vpeak--luminosity correlation
%	(C) amplitude and (D) functional form of photo-z errors in model (constrained by data correlation function)
%
%For each relevant simulation snapshot and corresponding photo-z bin:
%	For each TBD (sub)halo property for SHAM:
%		For each relevant data sample \neff (north/south and optical/IR parent sample):
%			For each TBD implementation of SHAM scatter:
%				select (sub)halo catalog and assign luminosities
%				If chosen photo-z implementation is TBD:
%					For each TBD photo-z error implementation option:
%						compute full sample clustering and compare to data (no luminosity bins)
%				Else:	
%					compute full sample clustering with chosen photo-z error implementation and compare to data
%
%				compute luminosity-binned clustering (with chosen photo-z error implementation) and compare to data
%				assign relevant color(s) to selected (sub)halo catalog in narrow (TBD) luminosity bins
%				compare model CMD to data
%				TBD: compute color-binned clustering (with chosen photo-z error implementation) and compare to data

%DATA CLUSTERING STRUCTURE:
%	BASEDIR /
%	clustering /
%	z-band apparent mag cut /	
%	r OR W1 absolute mag cut / 
%	full sample OR absolute mag bins / 
%	wp /
%	wp by field (north, south), photo-z bin, sample type (GxG, LxL, LxnL), pimax (<=150 Mpc/h)


%%%%%%%%%%%%%%%%%%%%%%%%%%%%%%%%%%%%%%%%
\section{}\label{}
%%%%%%%%%%%%%%%%%%%%%%%%%%%%%%%%%%%%%%%%


%%%%%%%%%%%%%%%%%%%%%%%%%%%%%%%%%%%%%%%%
\section{}\label{}
%%%%%%%%%%%%%%%%%%%%%%%%%%%%%%%%%%%%%%%%



%%%%%%%%%%%%%%%%%%%%%%%%%%%%%%%%%%%%%%%%
\section{Summary and Conclusions}\label{}
%%%%%%%%%%%%%%%%%%%%%%%%%%%%%%%%%%%%%%%%


\acknowledgements

%Legacy Surveys DR9

%Work done at Argonne National Laboratory was supported by the U.S. Department of Energy, Office of Science, Office of Nuclear Physics, under contract DE-AC02-06CH11357. We gratefully acknowledge use of the Bebop cluster in the Laboratory Computing Resource Center at Argonne National Laboratory. Computational work for this paper was also performed on the Phoenix cluster at Argonne National Laboratory, jointly maintained by the Cosmological Physics and Advanced Computing (CPAC) group and by the Computing, Environment, and Life Sciences (CELS) directorate.
%PB was partially funded by a Packard Fellowship, Grant \#2019-69646.

%Funding for SDSS-III has been provided by the Alfred P.\ Sloan Foundation, the Participating Institutions, the National Science Foundation, and the U.S.\ Department of Energy Office of Science. The SDSS-III website is http://www.sdss3.org/.
%SDSS-III is managed by the Astrophysical Research Consortium for the Participating Institutions of the SDSS-III Collaboration including the University of Arizona, the Brazilian Participation Group, Brookhaven National Laboratory, Carnegie Mellon University, University of Florida, the French Participation Group, the German Participation Group, Harvard University, the Instituto de Astrofisica de Canarias, the Michigan State/Notre Dame/JINA Participation Group, Johns Hopkins University, Lawrence Berkeley National Laboratory, Max Planck Institute for Astrophysics, Max Planck Institute for Extraterrestrial Physics, New Mexico State University, New York University, Ohio State University, Pennsylvania State University, University of Portsmouth, Princeton University, the Spanish Participation Group, University of Tokyo, University of Utah, Vanderbilt University, University of Virginia, University of Washington, and Yale University.

The CosmoSim database used in this paper is a service by the Leibniz-Institute for Astrophysics Potsdam (AIP). The MultiDark database was developed in cooperation with the Spanish MultiDark Consolider Project CSD2009-00064.
The authors gratefully acknowledge the Gauss Centre for Supercomputing e.V.\ (www.gauss-centre.eu) and the Partnership for Advanced Supercomputing in Europe (PRACE, www.prace-ri.eu) for funding the MultiDark simulation project by providing computing time on the GCS Supercomputer SuperMUC at Leibniz Supercomputing Centre (LRZ, www.lrz.de).
%The Bolshoi simulations have been performed within the Bolshoi project of the University of California High-Performance AstroComputing Center (UC-HiPACC) and were run at the NASA Ames Research Center.

\bibliography{refs}

\end{document}