
\documentclass[10pt, oneside]{article}   	% use "amsart" instead of "article" for AMSLaTeX format
\usepackage[top=1in, bottom=1in, left=1in, right=1in]{geometry} %, headsep=.15in, footskip=.3in]{geometry}
\geometry{letterpaper}                   		% ... or a4paper or a5paper or ... 
%\geometry{landscape}                		% Activate for for rotated page geometry
%\usepackage[parfill]{parskip}    		% Activate to begin paragraphs with an empty line rather than an indent
\usepackage{graphicx}				% Use pdf, png, jpg, or eps§ with pdflatex; use eps in DVI mode
								% TeX will automatically convert eps --> pdf in pdflatex
\input{../../tex/defs}								
										
\usepackage{xspace}
\usepackage{amssymb}
\usepackage{amsmath}

\usepackage{setspace}

\usepackage{natbib}
\setlength{\bibsep}{0pt}

\bibliographystyle{aasjournal}
%\usepackage[colorlinks,citecolor=blue]{hyperref}
%\usepackage[style=authoryear,natbib=true]{biblatex}

%\newcommand{\degsq}{\ensuremath{{\rm deg}^2}\xspace}
%\newcommand{\logm}{\ensuremath{\log(M_*/\msun)}\xspace}
%\newcommand{\mh}{\ensuremath{M_{\rm h}}\xspace}
%\newcommand{\msun}{\ensuremath{{\rm M}_\odot}\xspace}
%\newcommand{\Mpch}{\ensuremath{h^{-1}{\rm Mpc}}\xspace}
%\newcommand{\wprp}{\ensuremath{\omega_{\rm p}(r_{\rm p})}\xspace}

\newcommand{\aj}{AJ}
\newcommand{\apj}{ApJ}
\newcommand{\apjs}{ApJS}
\newcommand{\mnras}{MNRAS}


\title{Motivation for assessing biases of the DESI LRG sample with SHAM-based mocks + age matching}
\author{Angela Berti}
\date{}							% Activate to display a given date or no date

\begin{document}
\maketitle
%\section{}
%\subsection{}

%High level question:\ How can we make the LRG sample as useful as the BGS sample given its much larger volume?

%\section*{Motivation for SHAM + age/color matching-based mocks for the DESI LRG sample}

Galaxy redshift surveys reveal the large-scale structure of the universe and provide a means to study its expansion history. Large volumes are required to suppress statistical measurement errors due to both sample and cosmic variance, and are achieved with a combination of survey depth and large survey area. The Sloan Digital Sky Survey \citep[SDSS;][]{york_etal00} mapped approximately one-third of the entire sky to $z\sim0.2$ with spectroscopic redshifts, providing the first large-volume three-dimensional map of much of the local universe. Subsequent spectroscopic galaxy redshift surveys have probed much smaller areas (a few square degrees) to $z\sim1$ (e.g.\ PRIMUS \citep{coil_etal11, cool_etal13} and DEEP2 \citep{newman_etal13}), and are optimized for studying galaxy formation and evolution by obtaining statistically complete samples that include galaxies with stellar masses down to $\sim10^{9.5}\ \msun$, below the break in the galaxy stellar mass function.
%Other pencil-beam 0.2 < z < 1 spec-z surveys:%zCOSMOS (Lilly + (2007)
%VVDS (Le F�vre+ (2004, 2015)
%VIPERS (Guzzo+ (2014)

Bridging the gap between the large area of the SDSS and the depth of pencil-beam surveys like DEEP2 and PRIMUS are more recent large-volume spectroscopic and imaging surveys that sacrifice observing fainter and lower-mass galaxies in exchange for volumes large enough to constrain cosmological parameters, such as the dimensionless Hubble constant ($h$), the density of matter ($\Omega_{\rm m}$) and of energy ($\Omega_\Lambda$), and the amplitude of mass fluctuations on the scale of $8\ \Mpch$ ($\sigma_8$). The Baryon Oscillation Spectroscopic Survey \citep[BOSS;][]{dawson_etal13} obtained spectroscopic redshifts of 1.5 million galaxies with $\logm > 11$ to $z\sim0.7$ over 15.3 Gpc$^3$ (9,376 \degsq) to constrain cosmological parameters by measuring the baryon acoustic oscillation (BAO) signal present in the clustering of galaxies on scales of $\sim100\ \Mpch$.

%Hyper Suprime-Cam survey (HSC; Aihara+ (2018b)
%Dark Energy Survey (DES; Collaboration 2005)

The Dark Energy Spectroscopic Instrument \citep[DESI;][]{eisenstein_etal15} survey is spectroscopic galaxy redshift survey of unprecedented volume. Over the next five years DESI will obtain spectra for millions of galaxies in four target classes over 14,000 \degsq. The bright galaxy survey (BGS)  extends to $z\sim0.4$, while the the luminous red galaxy (LRG) sample will reach to $z\sim1$ and cover 20 times the volume of the BGS. The DESI survey will also observe emission line galaxies (ELGs) to $z\sim1.6$, and quasars to $z\sim3.5$.

DESI is fundamentally a BAO survey, and as such its target samples are optimized for precision measurements of cosmological parameters. The bright, massive galaxies DESI will observe trace the underlying dark matter distribution of the universe. However, DESI also offers novel opportunities for studying galaxy evolution, as well as constraining the high-mass end of the galaxy stellar mass function (SMF) and stellar-to-halo mass relation (SHMR), provided that sample selection effects are well-understood.

Unlike the essentially magnitude-limited BGS sample of relatively nearby galaxies, the DESI LRG sample is selected with a comparatively complex set of magnitude and color cuts, creating incompleteness that may depend on any combination of galaxy color, stellar mass, and redshift. However, the LRG sample covers a volume 20 times larger than that of the BGS sample, and will contain a much higher number density of spectroscopic redshifts at $0.3 < z < 1$ than any previous spectroscopic galaxy redshift survey, making it a good sample for statistical studies with negligible sample and cosmic variance errors. The rarity and associated low number density of massive galaxies $(\sim2\times10^{-5}\ {\rm Mpc}^{-3}$ for $\logm > 11.5)$ means that large volumes are essential for obtaining sample sizes large enough to for statistically significant measurements of the high-mass end of the galaxy SMF and the SHMR of LRGs.

\section*{Completeness studies of BOSS samples}

Numerous studies of the dependence of BOSS galaxy sample completeness on redshift and stellar mass inform how the same biases can be understood for the DESI LRG sample. BOSS contains two color- and magnitude-selected samples of massive galaxies $(\logm > 11)$:\ the LOWZ sample contains LRGs at $0.15 < z < 0.4(3)$, and the approximately stellar mass-limited CMASS (constant mass) sample, which selects galaxies of all colors at $0.4(3) < z < 0.8$.

Assessing the stellar mass completeness of BOSS samples requires stellar mass estimates for BOSS galaxies, as well as the total stellar mass function over the same redshift range and stellar mass range of the sample of interest. \citet{leauthaud_etal16} estimate stellar masses for the LOWZ and CMASS samples by fitting SEDs with linear combinations of \citet{bruzual_charlot03} stellar population templates. For the total galaxy SMF they use a combination of PRIMUS data for lower masses $(\logm \lesssim 11.3)$, and the Stripe 82 Massive Galaxy Catalog (S82 MGC) for the high-mass end of the SMF. Stellar mass estimates for both samples are from SED fitting techniques; \citet{moustakas_etal13} describes the derivation of the PRIMUS SMF, while S82 MGC estimates use the methods described in \citet{bundy_etal10, bundy_etal15}. S82 MGC is stellar mass-complete above $\logm \gtrsim 11.2$ \citep{bundy_etal15}, where the PRIMUS SMF is poorly constrained due to low number counts of massive galaxies (PRIMUS is $<10\ \degsq$).

To estimate the completeness of BOSS \citet{leauthaud_etal16} use the total galaxy SMF at $z\sim0.3$ for LOWZ and $z\sim0.55$ for CMASS. They find LOWZ to be 80\% complete at $\logm > 11.6$ over the full sample range of $0.15 < z < 0.43$. CMASS is 80\% complete at $\logm > 11.6$ *only* at $0.51 < z < 0.61$.

\citet{tinker_etal17} quantify the completeness of BOSS CMASS target in terms of the combined abundance of the CMASS, (CMASS\_)SPARSE, and WISE(\_COMPLETE) samples, limited to the peak in the CMASS redshift distribution at $0.45 < z < 0.6$. The main objectives of \citet{tinker_etal17} are to measure the SHMR of massive galaxies $(\logm \sim 11.5)$, and compare a variety of methods for estimating stellar mass, and as such they do not estimate the total galaxy SMF as in \citet{leauthaud_etal16}, although they do compare their results to the \citet{leauthaud_etal16} estimate. Although the relevant redshift ranges are not exactly the same (and neither are the denominators), \citet{tinker_etal17} estimate slightly higher completeness at all stellar masses:\ $\sim50\%$ at $\logm \sim 11.4$ versus the \citet{leauthaud_etal16} estimate of $\sim30\%$.

To characterize the SHMR for high-mass galaxies \citet{tinker_etal17} estimate the large-scale galaxy bias as a function of stellar mass for $\logm > 11.4$ using 12 different stellar mass estimates. They construct SHAM-based mocks with Rockstar \citep{behroozi_etal13b} catalogs for the MultiDark simulation $(\Omega_{\rm m}=0.27$ and $\sigma_8=0.82)$ by correlating stellar mass and halo mass, and add random Gaussian scatter of width $\sigma_{\log M_*}$ to galaxy stellar masses, where $\sigma_{\log M_*}$ derives from the relevant galaxy SMF (Wetzel \& White 2010). They also implement SHAM with luminosity and velocity dispersion for the sake of comparison with stellar mass.

\citet{tinker_etal17} fit galaxy bias versus stellar mass for a range of scatter amplitudes at fixed halo mass to constrain the scatter in the SHMR above $\logm \sim 11.4$. Their main results are
\begin{enumerate}
\item the \citet{chen_etal12} PCA-based stellar mass estimates have the tightest correlation with halo mass,
\item luminosity is *less* tightly correlated with halo mass than stellar mass regardless of the estimation method,
\item velocity dispersion correlates *as well as* photometrically-derived stellar masses, but worse than PCA stellar mass estimates, and
\item almost any stellar mass estimate offers a more robust rank-ordering of a galaxy sample with a heterogeneous color distribution than does luminosity.
\end{enumerate}
\noindent They also find that the bias of CMASS galaxies is minimized at $\logm \sim 11.4$, with larger bias at both lower and higher stellar masses, and note that the increase in bias at lower stellar masses could be attributable to the CMASS selection cuts preferentially selecting satellites at this mass range.

\citet{montero-dorta_etal16b} measure the luminosity function and completeness versus redshift, color, and $i$-band magnitude of the red sequence (RS) component of CMASS at $z\sim0.55$. I don't fully understand their method, but they claim to deconvolve the intrinsic CMASS color--color and color--magnitude distributions from photometric errors and selection effects, which involves modeling the covariance matrix for $i$-band magnitude, ${g-r}$ color, and ${r-i}$ color with Stripe 82 data, and doesn't rely on any SPS models. They find the completeness of the RS as a function of $i$-band luminosity is 100\% at $i\sim19.5$ and falls to zero at $i\sim20.0$ for $0.5 < z < 0.7$. The completeness with redshift is $\sim30\%$ at $z\sim0.45$ and rises to a plateau of $\sim80\%$ at $z\sim0.575$.

%Saito+ 2016 (https://doi.org/10.1093/mnras/stw1080)
%Sec. 5.2, 5.3 for SHAM implementation based on Hearin+ (2013 methods for SDSS
%H13 SHAM implementation at lower stellar masses:\ z_starve dominated by z_form
%Saito's implementation of SAM with CMASS:\ z_starve dominated by z_char
%z_form component of z_starve causes red galaxies to cluster *less* strongly than blue
More relevant for the work I'm envisioning is \citet{saito_etal16}, which uses SHAM and a simple version of age matching to create $z\sim0.5$ mocks with and without added assembly bias effects. They also explicitly do not assume a constant CMASS HOD with redshift or galaxy color, although they do assume a constant total SMF over $0.43 < z < 0.7$.

For the total galaxy SMF \citet{saito_etal16} follow a method similar to \citet{bundy_etal15}, who use \citet{bundy_etal10} SED stellar mass estimates for S82 MGC to compute the SMF above $\logm > 10.5$ over $0.43 < z < 0.7$. \citet{saito_etal16} mention PRIMUS for constraining the lower-mass end of the SMF (as in Leauthaud et al.), but it's not clear to me whether they actually use PRIMUS or just extrapolate the full SMF from S82 MGC the mass range where it's complete.

For SHAM \citet{saito_etal16} use Rockstar halo catalogs for the MultiDark simulation \citep[MDR1;][]{riebe_etal13}. Their model predicts that their observed increase in the stellar mass of the CMASS sample with redshift by a factor of 1.8 should increase the clustering by a factor of 1.5 (and mean halo mass by a factor of 3.5) over the same redshift range, and they emphasize that this result is independent of the choice of the halo parameter used for SHAM. They note that \citet{reid_etal14} observed a constant clustering amplitude with redshift for CMASS and thus assumed a redshift-independent HOD, and conclude that the \citet{reid_etal14} result may be due to variation in color and stellar mass over the same redshift range that coincidentally compensate for each other to produce a constant clustering amplitude. Specifically, they note the CMASS selection function excludes recently star-forming galaxies at low redshift $(z < 0.6)$ and fixed stellar mass, while at higher redshift $(z > 0.6)$ the CMASS sample is flux-limited and therefore includes a larger range of galaxy colors at fixed magnitude. Additionally, at least at lower masses redder galaxies cluster more strongly than bluer galaxies at fixed stellar mass. If this trend persists at higher masses, then the inclusion of bluer galaxies in CMASS at higher redshifts might offset the effect of increased mean stellar mass on the clustering signal. A primary conclusion of \citet{saito_etal16} is that assembly bias does factor into the galaxy--halo connection for high-mass galaxies, i.e.\ the SHMR for these galaxies has some dependence on galaxy color, and should not be inferred from the clustering signal without any consideration of color.

%\citet{guo_etal18} (https://doi.org/10.3847/1538-4357/aabc56)
%Validate method against mock catalogs with three different assumed completeness functions and successfully recover the input completenesses
%Find slope of SHMR steepens with increasing redshift over BOSS redshift range
\citet{guo_etal18} actually test how stellar mass incompleteness affects the inferred SHMR of BOSS galaxies. They also model completeness versus stellar mass and redshift separately for centrals and satellites. Agreement with \citet{leauthaud_etal16} (comparing the Guo et al.\ model for centrals to that of Leauthaud et al.\ for all galaxies) is generally quite good, except for $\logm > 11.6$ at $0.6 < z < 0.7$, where Guo's central completeness plateaus at $\sim60\%$ but the Leauthaud et al.\ estimate is close to unity.

Their analysis is motivated (at least in part) by the differences between the color distributions of centrals and satellites \citep[e.g.][]{yang_etal18}, and they ask whether this causes different incompleteness effects between centrals and satellites in BOSS samples. They add an incompleteness component to the \citet{yang_etal12} conditional stellar mass function (CSMF) model with the goal of predicting the \emph{total} galaxy SMF and SHMR from incomplete BOSS samples (ICSMF model).

The \citet{yang_etal12} CSMF assumes the average number of central galaxies of stellar mass $M_*$ hosted by halos of mass \mh is a lognormal function with scatter $(\sigma)$ that depends on halo mass (versus, say, the \citet{tinker_etal17} estimate of $\sigma \sim 0.18$ for CMASS). \citet{guo_etal18} assume constant $\sigma=0.173$ over $0.1 < z < 0.8$, and model the average central galaxy stellar mass in halos of mass \mh as a broken power law (i.e.\ different slopes for the low-mass and high-mass ends). They then model the completeness of BOSS, $c(M_*)$, in seven redshift bins with the functional form \citet{leauthaud_etal16} find fits their completeness estimates (an error function with three free parameters), but \citet{guo_etal18} allow for different parameter values for the completenesses of centrals and satellites (six free parameters). This gives their observed (i.e.\ incomplete) BOSS galaxy SMF model 12 total parameters, four for the assumed galaxy SMF and three each for the central and satellite completeness functions). They constrain these parameters by comparing the predicted incomplete SMF and projected clustering to the data using an MCMC method, and then apply the best-fit parameters to the assumed intrinsic galaxy SMF (same as the 12-parameter model but without the $c(M_*)$ factors in the SMF integrands). This method is completed separately for each of seven redshift bins over the full BOSS sample range, using a mock constructed with the BigMDPL simulation \citep{klypin_etal16} snapshot closest to the median the redshift of each bin ($z=0.152$ to 0.759)

There are several important caveats to the \citet{guo_etal18} ICSMF model:\ (1) assumes the same SHMR for centrals and satellites; (2) ignores possible evolution of the CSMF between accretion and observation redshifts (i.e.\ uses $M_{\rm acc}$ for satellite mass); and (3) similarly ignores possible growth and/or tidal stripping of satellites after accretion. They don't adopt different central and satellite SHMRs because adding four additional model parameters would cause overfitting as the central and satellite SHMRs are not independent \citep{guo_etal16}, and directly addressing these caveats is left to future work.

\citet{guo_etal18} acknowledge their method assumes incompleteness is uncorrelated with galaxy color, which isn't necessarily true, especially for color-selected samples over a wide redshift range. They test this assumption with a ${g-i>2.35}$ color cut on the BOSS sample at $0.5 < z < 0.6$ to remove blue galaxies (21\% of the sample) within the limits of photometric errors. Results are strikingly similar for the full BOSS sample versus the strict red color cut, with the exception of the satellite galaxy completeness at $\logm > 11.2$, which drops by roughly 10\% from the full BOSS sample. Based on this test \citet{guo_etal18} conclude that ignoring galaxy color in their analysis is inconsequential because red galaxies dominate the high-mass end of the total galaxy SMF, although they also assert that differences in the SHMRs of blue and red galaxies likely can't be ignored at lower stellar mass.

\citet{bates_etal19} estimate the completeness of BOSS samples using a technique (``clustering redshifts") that involves computing the angular cross-correlation of a photometric (``unknown") redshift sample with a spectroscopic (?reference") sample in different redshift bins to obtain a redshift distribution for the photometric sample. This method also requires an estimate of the galaxy bias evolution, which \citet{bates_etal19} obtain from both the LGalaxies SAM \citep{henriques_etal15} with the Millennium simulation and the SAGE analytic model \citep{croton_etal16} with the MDPL2 simulation. Their reference sample is the combined BOSS LRG, eBOSS LRG, and eBOSS QSO samples, and their unknown sample is SDSS DR8 galaxies with ${i < 21}$ (for completeness), masked to the region of overlap with the reference sample.

I don't fully understand the motivation for using this method to estimate the redshift distribution of SDSS galaxies from photometry; I suppose it's mainly a proof-of-concept exercise. Regardless, they construct SMFs and $i$-band luminosity functions for SDSS galaxies in narrow bins of color and magnitude, then use the clustering redshifts method to recover the redshift distribution of each bin. Galaxy stellar masses and luminosities are inferred from the combination of the SMF, luminosity function, and redshift distribution of each bin, plus comparisons with mock data. They estimate BOSS completeness using their inferred total SMF to $z\sim0.8$, and find it to be $\sim80\%$ above $\logm > 11.4$ at $0.2 < z < 0.7$, but only $\sim30\%$ at $0.7 < z < 0.8$ at the highest masses $(\logm > 11.6)$.

\section*{Summary of existing work on the ``lensing is low" discrepancy}

\citet{leauthaud_etal17} compare clustering and galaxy-galaxy (g-g) lensing measurements of BOSS CMASS galaxies to 
predictions from mocks tuned to match the observed CMASS SMF and clustering \citep{saito_etal16, rodriguez-torres_etal16}. To account for stellar mass incompleteness in mocks they downsample mock galaxies according to stellar mass to match the redshift-dependent CMASS SMFs; these mocks do *not* account for color incompleteness of CMASS sample.
\citet{leauthaud_etal17} find standard galaxy--halo connection models (both SHAM and HOD models) that match the observed CMASS clustering predict a g-g lensing signal 20 to 40\% larger than the observational result. They leave tests of models with galaxy assembly bias to potentially resolve the ``lensing is low" discrepancy to other papers (summarized below), but do assess the impact of baryonic effects on CMASS-like samples with the Illustris \citep{vogelsberger_etal14, nelson_etal15} simulations. Comparing results from full hydrodynamical and dark matter only versions of Illustris, they find that that baryonic effects can induce a 10\% to 30\% difference in the lensing metric $\Delta \Sigma$ (on one-halo scale), and a factor of two difference in the satellite fraction for galaxy samples with CMASS-like number densities.

Two conclusions from the ``lensing is low" paper stand out as particularly relevant for my work:

\begin{quotation}
``In \citet{saito_etal16}, we present the first analysis of the effects of assembly bias on the clustering properties of CMASS. However, our analysis assumed a simplified model for the color completeness of CMASS. To build on \citet{saito_etal16}, the next step would be to characterize the color-completeness of CMASS and to explore the impact of assembly bias using, for example, conditional SHAM techniques \citep[e.g.][]{hearin_etal14}."
\end{quotation}

This is essentially a high-level description of what I'm working on:\ incorporating age-matching at multiple redshifts into an analysis similar to that of \citet{saito_etal16}. The second conclusion from \citet{leauthaud_etal17} that motivates my project, particularly in the context of DESI, is:

\begin{quotation}
``If the ages of galaxies correlate with the ages of their dark matter halos, then assembly bias effects may be present in color-selected samples such as CMASS. The clustering of CMASS tightly constrains the large-scale bias of the sample whereas the lensing is sensitive to the mean halo mass. The difference that we observe may suggest a tension between the halo mass and the large-scale bias of this sample---the smoking gun for assembly bias. If assembly bias is at play, it could be a systematic effect for RSD constraints from upcoming surveys such as DESI \citep{levi_etal13}."
\end{quotation}

\citet{zu20} claim to largely reconcile the ``lensing is low" discrepancy on small scales $(r\lesssim1\ \Mpch)$ by extending the iHOD framework of \citet{zu_mandelbaum15, zu_mandelbaum16} to include halo mass dependence of the satellite selection function. Their model lensing g-g predictions are still mostly $\sim30\%$ higher than the observed signal at larger scales in all three redshift bins (bin edges are 0.43, 0.51, 0.57, 0.7).

They directly measure the selection function of LOWZ centrals by comparing the sample to the brightest central galaxies (BCGs) of redMaPPer \citep{rykoff_etal14} cluster catalogs, and find it to be largely independent of halo mass (at least in the cluster mass range of redMaPPer). Satellite completeness is modeled as a joint function of stellar and halo mass using the erf functional form of \citet{leauthaud_etal16}. \citet{zu20} find the satellite detection fraction declines from high completeness at high stellar mass to zero at low stellar mass, with the characteristic stellar mass (separating low- and high-mass regimes) and the slope of decline both dependent on host halo mass. Specifically, satellite selection is independent of halo mass at $\mh>10^{13.76}\ \msun$, reflecting the convergence of satellite colors to the red sequence for the most massive host halos.

The interpretation of these results is that the BOSS LOWZ magnitude and color cuts select more low-mass satellites in high-mass halos, while CMASS selection skews toward low-mass satellites in less massive halos, especially at higher redshifts. This is consistent with LOWZ selection isolating LRGs and CMASS selection including more blue galaxies.

\citet{zu20} speculate that the low g-g lensing signal could be caused by BOSS CMASS galaxies residing in lower concentration halos, and/or not accounting for the dependence of the BOSS detection fraction at fixed stellar mass on halo properties that correlated with galaxy colors. This assessment is in qualitative agreement with the tests of assembly bias done by \citet{lange_etal19, lange_etal21a} and the concentration-based assembly bias models of \citet{yuan_etal20, yuan_etal21} described below.

\citet{lange_etal19} compare fits of BOSS data using a standard HOD and two decorated HOD (dHOD) models with an added assembly bias parameter (NFW concentration, $c_{\rm NFW}$, and halo spin, $\lambda$) in ``low" $(11 < \logm < 11.5)$ and ``high" $(11.5 < \logm < 12)$ stellar mass bins. They find that some form of assembly bias could *partially* account for the ``lensing is low" discrepancy, but only up to $\sim10\%$ (i.e.\ one quarter to one half of the observed 20--40\% difference), and note that \wprp alone does not constrain the values of the assembly bias parameters in their dHOD models.

They also test baryonic effects with the Illustris and TNG300 simulations, and find---as with they assembly bias models they test---that baryonic effects can account for at most 10\% of the discrepancy. \citet{lange_etal19} conclude that a combination of baryonic feedback effects, assembly bias, and possible cosmological parameter revision may be required to fully resolve the ``lensing is low" discrepancy; and suggests a more comprehensive study of the impact of galaxy assembly bias is required.

In a related paper \citet{lange_etal21a} study the radial and stellar mass dependence of the ``lensing is low" discrepancy using the same general analysis approach as \citet{leauthaud_etal17}. They compare three stellar mass estimates for the LOWZ sample:\ ``Wisconsin" PCA, ``Granada" photometry, and cross-correlations between DESI Legacy Imaging DR8 photometry and S82 MGC galaxies with spectroscopic redshifts, and find a lensing amplitude mismatch of $\sim35\%$ independent of both stellar mass (and also therefore of halo mass in $10^{13.3}$ to $10^{13.9}\ h^{-1} \msun$ range) and radial scale.

\citet{yuan_etal20} test extended HOD models with added satellite distribution and satellite velocity bias parameters, constrained with \wprp, and a halo concentration assembly bias parameter. They focus on the CMASS sample, and their fits still yield a $\sim34\%$ ``lensing is low" discrepancy.

In \citet{yuan_etal21} they test extended HOD models with an environment-based assembly bias term $(A_{\rm e})$ and a halo concentration-based term $(A)$.  They find including $A$ and constraining parameters with the redshift-space two-point correlation function (2PCF) instead of \wprp decreases the average halo mass per galaxy by 12\%, relative to a standard five-parameter HOD, which is strongly rejected. Including $A_{\rm e}$ in their model decreases the average halo mass by 10\%, while including both $A$ and $A_{\rm e}$ and constraining with the redshift-space 2PCF predicts a 26\% decrease in average halo mass. This translates to a g-g lensing signal prediction that agrees with observation within $1\sigma$, and \citet{yuan_etal21} conclude that an environmental assembly bias term is a ``physical and indispensable addition to the HOD."

\section*{Preliminary modeling of the DESI LRG sample}

The stated goal of \citet{zhou_etal20} is to improve HOD fits to the DESI LRG sample that use photometric redshift clustering measurements, and as such they don't consider the incompleteness of the LRG sample.
\citet{zhou_etal20} fit \wprp measured in five redshift bins from $0.4 < z < 0.9$ with a five-parameter HOD model, and constrain HOD parameters with mocks constructed from MDPL2 simulation snapshots near redshift bin centers populated with galaxies as in \citet{zheng_etal07}.
They find similar HOD parameters at $0.4 < z < 0.8$, and statistically significant differences in model parameters for only the highest redshift bin $(0.8 < z < 0.9)$. This is consistent with a constant clustering amplitude for LRGs over the DESI LRG sample redshift range, but without considering incompleteness this result isn't sufficient to draw definitive conclusions about the (lack of) redshift evolution of the intrinsic LRG SHMR.

\citet{hernandez-aguayo_etal21} supports DESI LRG modeling beyond the HOD framework. Their focus is on large-scale galaxy clustering and generating large numbers of mocks for accurate clustering covariance measurements, but the conclusions support more nuanced galaxy--halo connection modeling than the standard HOD.

They apply the {\sc Galform} \citep{cole_etal00} semi-analytic model (SAM) to the Planck-Millennium simulation, which they claim has sufficient resolution to make robust predictions for galaxy masses down to $10^7\ h^{-1}\msun$. {\sc Galform} predicts absolute magnitudes with dust attenuation, which are then converted to apparent magnitudes at various redshift snapshots, enabling DESI LRG selection cuts to be applied to the mocks.
%underpredicts LRG abundance relative to estimates from Zhou+ (2020) photo-zs

The most relevant conclusions from \citet{hernandez-aguayo_etal21} are:
\begin{enumerate}
\item
The halo occupation of central galaxies does *not* reach unity for the most massive halos and drops with increasing mass, indicative of a non-trivial LRG--halo connection that is not modeled well with a standard HOD.
\item
DESI LRG selection criteria excludes a small but important fraction of the most massive galaxies ($\logm > 11.15$).
\item
Stellar mass is an insufficient proxy to select LRGs; all bright galaxies in the $W1$-band luminosity function are LRGs, compared to only about half of the galaxies at the bright end of the $r$-band luminosity function.
\item
Comparing the HOD and subhalo mass functions of stellar mass-selected galaxies with those of LRG samples shows the DESI LRG selection cuts affect the selection of subhalos populated by LRGs; (sub)halo mass alone is not sufficient to determine whether a subhalo hosts an LRG.
\end{enumerate}


\section*{Motivation for SHAM + age matching modeling of DESI LRGs}

\begin{itemize}
\item Smaller volume simulations ($\sim1~\Gpch$) with dark matter particle mass resolution sufficient for subhalo detection and tracking across snapshots enables small-scale clustering analysis that can't be done with large-volume mocks for BAO studies.
\item Galaxy luminosity and especially color is difficult to model analytically; SHAM allows luminosity functions and color distributions from photometry to be reproduced for (i.e., mapped onto) halo distributions with precisely known statistics (from $N$-body simulations).
\item Standard HOD analytic models require assumptions about functional form and number of free parameters, and don't necessarily account for (unknown) incompletenesses in the galaxy samples to which they are applied.
\item Empirical modeling provides a way to quantify incompleteness, satellite fraction, etc.\ of LRG sample using the full galaxy population (DESI Legacy Survey photometry).
\item Similar method has been used to model DESI-like BGS sample \citep{safonova_etal20}.
\end{itemize}


\small
\bibliography{/Users/aberti/Desktop/research/tex/refs}

\end{document}

%Saito+ 2016
%NOTES in odt doc

%Safonova+ 2020 DESI BGS SHAM mock
%NOTES in odt doc

%DeRose+ 2021
%NOTES in odt doc

%Reid+ 2014 (https://doi.org/10.1093/mnras/stu1391)
%model dn/dz by downsampling redshift-independent HOD
%assume CMASS is single, homogeneous sample (so don't assume sample is complete, but do assume representative downsampling, which is questionable)
%use of non-evolving HOD motivated by observation that CMASS clustering does not vary strongly w/ redshift
%random downsampling does not modify clustering, so R14 find constant clustering amplitude (and therefore constant halo mass) w/ redshift

%Leauthaud+ 2016 (https://doi.org/10.1093/mnras/stw117)
%The Stripe 82 Massive Galaxy Project---II.\ Stellar mass completeness of spec-z BOSS samples
%birth parameter b_1000:\ ratio of average SFR within past 1 Gyr to SFR average over galaxy's history
%b_1000 estimates from Blanton & Roweis 2007 KCORRECT package
%estimate BOSS completeness using analysis from other studies
%Miyatake+ (2015) constrain high-mass end of SHMR via HOD model with subsamples of CMASS
%More+ (2015) assume incompleteness is random downsampling
%Shankar+ (2014)
%HOD modeling of wp of CMASS at 0.4 < z < 0.6 and 0.6 < z < 0.8
%impose stellar mass cut log M* > 11.5 on both redshift samples
%assume above samples are complete for sake of HOD analysis

%Bundy+ (2017) (https://doi.org/10.3847/1538-4357/aa9896)
%The Stripe 82 Massive Galaxy Project---III.\ A lack of growth among massive galaxies
%construct mass-limited sample of S82-MGC galaxies at log M* > 11.2
%55\% spec-zs for sample
%41,770 objects
%measure SMF over 0.3 < z < 0.65
%find no growth in typical stellar mass of massive galaxies over redshift interval

%Stoppacher+ 2019 (https://doi.org/10.1093/mnras/stz797)

%Lange+ 2021 (https://arxiv.org/pdf/2101.12261.pdf)
%BOSS LOWZ simulation-based RSD analysis


%Dey+ 2019 (https://doi.org/10.3847/1538-3881/ab089d)
%Overview of DESI Legacy Imaging Surveys


%SHAM techniques

%Hearin+ 2013 (https://doi.org/10.1093/mnras/stt755)
%SHAM beyond clustering: new tests of galaxy-halo abundance matching with galaxy groups

%Hearin & Watson 2013 (https://doi.org/10.1093/mnras/stt1374)
%Dark side of galaxy color
%Primary calculation of z_starve for abundance matching

%Hearin+ 2014 (https://doi.org/10.1093/mnras/stu1443)
%Dark side of galaxy color: evidence from new SDSS measurements of galaxy clustering and lensing

%Campbell+ 2015 (https://doi.org/10.1093/mnras/stv1091)
%Assessing color-dependent occupation statistics inferred from galaxy group catalogs

%Lehmann+ 2017 (https://doi.org/10.3847/1538-4357/834/1/37)
%Concentration dependence of galaxy-halo connection:\ modeling assembly bias w/ abundance matching
%Create mocks via abundance matching as in Behroozi+ (2010) and Reddick+ (2013)
%deconvolve scatter from luminosity function
%abundance match luminosity with halo proxy
%replace scatter by adding log-normal scatter to galaxy catalog
