\documentclass[11pt,preprint]{aastex}

\topmargin=-0.5in
\setlength{\textwidth}{6.5truein}
\setlength{\textheight}{9truein}
\pagestyle{myheadings}
\pagenumbering{arabic}

\input{../../tex/defs}

\begin{document}

\thispagestyle{plain}
\markright{Referee report response:\ The Galaxy--Halo Connection of DESI LRGs with SHAM}

\begin{flushleft}
To the Referee:

We thank the referee for their thoughtful report, which has led to a clearer presentation of our results.\ Below we have written a response to each comment.\ For reference we include the referee's original comments in their entirety in blue text, while our responses appear in black text.\ Changes and additions appear in the manuscript in {\color{red} red} text.

In addition to changes made in response to the referee's comments, we have also modified Eq.\ 7 to accurately reflect the calculation of the projected correlation function as a sum over bins in line-of-sight separation from $-\pimax$ to \pimax.

\end{flushleft}
\noindent Sincerely,

\noindent Angela Berti
\vspace{1cm}

\noindent \reftxt{
I have a few queries regarding this definition of the ``starvation redshift", which is used as the proxy for halo age.\ Firstly, ``z\_acc" is defined as the time when a subhalo accretes onto a parent halo--is this simply set by when it crosses the host's R\_vir, or as the time when the halo shows a decrease in mass from its peak mass (in the assembly history)? I am next curious about the quantity ``M\_char", which is set at $10^{12}$ Msun/h based on Hearin \& Watson (2013).\ What sets this value? Is it purely empirical or there a physical interpretation as to why this mass scale is relevant (such as e.g.\ being roughly the regime beyond which AGN feedback starts to become important)? While I understand that the specific value might not affect the conclusions in any significant way, it would be helpful if the authors could elaborate a little.}

We have added the following to \S2.1.2 to clarify the meaning of \zacc:
\begin{quotation}
Hearin \& Watson (2013) follow Behroozi et al. (2012c), which defines $z_{\rm acc}$ as the snapshot after which a subhalo always remains a subhalo. They note that alternative definitions, such as that of Wetzel et al. (2014), where $z_{\rm acc}$ is the snapshot at which a subhalo has been identified as such for two consecutive snapshots, have little impact on their results.
\end{quotation}

We have also expanded the description of \mchar in \S2.1.2 to the following:
\begin{quotation}
We adopt the value of \mchar =  $10^{12}$ \msun/$h$ used in Hearin \& Watson (2013), who note that their results are insensitive to the precise value of \mchar used. The empirical and physical motivation for M\_char are described in detail in \S6.3 of Hearin \& Watson (2013). Briefly, there is empirical support for a characteristic halo mass above which star formation is highly inefficient:  $\sim10^{12}$ \msun/$h$ is the halo mass at which the SHMR peaks, falling off rapidly at higher halo masses (Behroozi et al. 2013; Yang et al. 2012, 2013; Moster et al. 2013; Watson \& Conroy 2013), and Behroozi et al. (2012a) have shown that this mass remains essentially constant throughout much of cosmic history.
\end{quotation}

\noindent \reftxt{
Is the shaded are beneath the curves in Fig.\ 1 representative of something? (I understand that the upper limit is set by value of z\_starve for that halo's accretion history).}

We have added the following to the caption: ``The shaded region below each curve corresponds a halo's mass accretion history before it reaches \zstarve."

\noindent \reftxt{
In calibrating the parameters of the model, the projected correlation function, w\_p, of DECaLS galaxies is used (Eq.\ 6).\ I am curious to know how different the conclusions of this paper might look had the redshift-space CF been used instead.\ The redshift-space version, by virtue of including galaxy velocities, will likely lead to a different galaxy-halo connection that incorporates the velocity dispersion of subhalos--presumably this additional information would enable a stronger constraint on which subhalos host LRGs? Are the uncertainties due to photometric redshift errors too big an issue to enable such an analysis?}

Yes, the typical uncertainty of photometric redshifts used for this work is on the order of several thousand km/s, which is too large with respect to typical subhalo velocity dispersion of ~500 km/s to enable this type of analysis.

\noindent \reftxt{
In Figs.\ 4 and 5, I note that the mock catalogs systematically underpredict w\_p (for r\_p $<$ 1 Mpc/h) compared to both the z- and W1-band parent galaxy samples.\ This seems most apparent in the lowest magnitude bin.\ I assume these scales (i.e.\ the one-halo regime) was also used in the fitting? Perhaps the authors could elaborate on the reason why this is further.}

We have added the following to the second to last paragraph in \S3.3 to emphasize that the shaded regions in Figures 4 and 5 ($\rp < 0.1\ \Mpch$) were not used for model fitting:
\begin{quotation}
Note that the shaded regions in Figures 4 and 5 at $\rp < 0.1\ \Mpch$ denote measurements not used for model fitting.
\end{quotation}

\noindent \reftxt{
At the end of Section 3.4, the authors come to the conclusion that the clustering predicted in the mocks suggests little correlation between z\_starve and color.\ What if another proxy for halo age--such as the redshift at which 50\% of the final-day mass is assembled, or indeed halo concentration--is used instead? The latter, in particular, would be interesting to consider.\ As the authors suggest later on, one reason halo and galaxy properties may decorrelate is to to the effects of feedback at late times.\ Hydrodynamical simulations suggest that the growth (and, consequently, the effect) of SMBHs is connected with the binding energy of the host DM halo (see, e.g., arXiv:1908.1138).}

Given that assigning mock colors at random (i.e.\ with no correlation between galaxy color and a halo age proxy) still generally overpredicts the clustering of LRGs relative to the data (or has no effect in the case of optical LRGs in the IR model, where the prediction already matches the data on two-halo scales), it is difficult to see how a different age proxy could result in predictions that are closer to the data. Additionally, halo concentration is already factored into the definition of \zstarve via \zform (Eq.\ 3).

\noindent \reftxt{
LRG-like populations have sometimes been defined by simply rank-ordering objects by halo/stellar mass to some fixed number density to represent the ``brightest/reddest galaxies".\ This clearly misses many of the nuances of the DESI target selection as performed in this paper.\ In this respect, it would be interesting to see in Fig.\ 10 what the predicted r\_p x w\_p would be when simply ranking subhalos by V\_peak until one reaches the same number density of LRGs.\ Can such a curve be added to this Figure (or discussed in the text)? I think it would be useful for the modeling community.}

We performed two tests using the $z_{\rm sim}=0.523$ redshift bin. The first, shown in Fig.~\ref{fig:test1} below, takes the magnitude-limited mock galaxy samples, then identifies LRGs by selecting the $N$ brightest mock galaxies (i.e.\ those with the largest values of \vpeak, with scatter) until the number density of the relevant DESI LRG sample (IR or optical) is reached.
\begin{figure}[h!]
\centering
  \includegraphics[width=0.8\linewidth]{../figures/wp_data-vs-mock/sham_brightest_only.png}
  \caption{Dashed lines show predicted mock LRG clustering if the DESI LRG target selection is ignored, and mock LRG samples are instead taken to be the $N$ brightest mock galaxies from the magnitude-limited samples up to recreating the number density of DESI LRGs (IR or optical). Solid lines are as in the middle column (i.e.\ $z_{\rm sim}=0.523$) of Figure 10 of the manuscript.
  \label{fig:test1}
  }
\end{figure}

We think that adding additional curves to Figure 10 would make this already busy figure too complex, and potentially detract from the comparison of the predictions of the ``default" and ``random colors" models. We have however added the following discussion to \S4.1:
\begin{quotation}
We also tested using proxies for IR and optical LRGs by selecting the $N$ brightest mock galaxies from the relevant magnitude-limited mock galaxy catalog to recreate the number densities of the IR and optical DESI LRG target samples, instead of applying the DESI LRG target selections to the magnitude-limited mocks as described in \S3.5. The result was a slightly larger predicted clustering amplitude compared to our ``random colors" models that do use the DESI LRG target selections, shown in Figure 10.
\end{quotation}

The second test, shown in Fig.~\ref{fig:test2} below, creates mock LRG samples by using the luminosity functions of DESI LRG samples for SHAM, instead of applying the DESI LRG target selection to magnitude-limited mock galaxy catalogs. We recognize that this method still uses the DESI LRG target selection, albeit in a different way than our main analysis.
\begin{figure}[h!]
\centering
  \includegraphics[width=0.8\linewidth]{../figures/wp_data-vs-mock/sham_lrg_only.png}
  \caption{Dashed lines show mock LRG clustering if SHAM is performed using the luminosity functions of just DESI LRG samples (IR or optical), instead of magnitude-limited samples from which LRGs are then selected using the DESI LRG target selection. Solid lines are as in the middle column (i.e.\ $z_{\rm sim}=0.523$) of Figure 10 of the manuscript.
  \label{fig:test2}
  }
\end{figure}

\noindent \reftxt{
In a similar vein, what does the radial distribution of satellite LRGs look like in the models corresponding to Fig.\ 10? In particular, are they representable by halo NFW profiles (with some modification)?}

The radial distribution of mock satellite LRGs comes directly from the MDPL2 DM simulation we used. However we have checked these distributions for each redshift bin, LRG selection (IR and optical), and model ($W1$ and $z$-band), and compared to an NFW profile with $R_{\rm s}=R_{\rm vir}$. Fig.~\ref{fig:sat_dist} below shows this for the $W1$-band model at $z_{\rm sim}=0.523$, and the satellite LRG radial distribution does roughly follow the shape of an NFW profile, where we have chosen an arbitrary value of $\rho_0$ to approximately match the normalization of the satellite distributions.
\begin{figure}[h!]
\centering
  \includegraphics[width=0.5\linewidth]{../figures/sat-dist_MW1_zsnap0p52323_NFWcomp.png}
  \caption{Radial distributions of mock LRG satellites for the $W1$-band model at $z_{\rm sim}=0.523$, compared to an NFW profile dark matter density profile.
  \label{fig:sat_dist}}
\end{figure}

\end{document}
